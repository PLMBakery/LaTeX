\section{Readme und Erklärungen}

\begin{enumerate}
	\item Entwickle einen Testplan: Definiere den Umfang, die Ziele, Ressourcen und den Zeitplan deines Testprojekts. Schließe spezifische Testfälle, Skripte und Kriterien für die Bewertung der Ergebnisse ein. Erstelle ein Dokument in ClickUp, um deinen Testplan zu entwerfen und zu dokumentieren.
	\item Erstelle Testfälle: Testfälle sind einzelne Tests, die durchgeführt werden sollten, um zu überprüfen, ob das Produkt oder die Software wie erwartet funktioniert. Erstelle Testfälle für jedes Feature oder jede Komponente des Produkts und nutze benutzerdefinierte Felder in ClickUp, um diese zu erstellen und Teammitgliedern zuzuweisen.
	\item Weise Aufgaben Testern zu: Sobald du deine Testfälle hast, weise Aufgaben deinen Testern zu. Verwende das Template, um Aufgaben zuzuweisen, den Fortschritt zu verfolgen und Ergebnisse zu dokumentieren. Nutze die Board-Ansicht in ClickUp, um Aufgaben Teammitgliedern zuzuweisen und deren Fortschritt zu verfolgen.
	\item Führe die Tests aus: Nachdem die Aufgaben zugewiesen wurden, ist es Zeit, die Tests durchzuführen. Deine Tester werden die Testfälle verwenden, um zu überprüfen, ob das Produkt wie erwartet funktioniert. Verwende den Gantt-Chart in ClickUp, um einen Zeitplan für die Tests zu erstellen und Fristen für jede Aufgabe festzulegen.
	\item Überprüfe und analysiere die Ergebnisse: Nachdem die Tests abgeschlossen wurden, überprüfe die Ergebnisse und analysiere eventuelle Abweichungen. Verwende die Testergebnisse, um Änderungen oder Anpassungen am Produkt vorzunehmen. Erstelle Meilensteine in ClickUp, um den Fortschritt zu verfolgen und die Ergebnisse zu überprüfen
\end{enumerate}
\section{Test Bild einbinden}
\setlength{\columnwidth}{15cm} %wenn das bild größer wie 10cm sein soll, muss das verwendet werden. Da sonst Fehlermeldungen kommen im LOG
\centering
\includegraphics[width=14cm,frame=0.25pt,vspace=1cm]{pics/logo}\captionof{figure}{Bild 1}
\subsection{Test Bild einbinden links rechts}
\begin{eintrag2}{
		\includegraphics[width=7cm,frame=0.25pt]{pics/logo}\captionof{figure}{Bild mit Text rechts}} 	hier steht der text neben dem Bild
\end{eintrag2}
\section{Abkürzungen verwenden}

Der\ac{eBOM} ist ein wichtiger Bestandteil des Produktentwicklungsprozesses. Die Beschreibung  \href{https://www.namsu.de/Extra/pakete/Acro.html}{ ist hier zu finden}