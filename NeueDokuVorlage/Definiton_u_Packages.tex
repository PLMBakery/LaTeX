% hier wird das Aussehen des Dokumentes definiert und die Pakete die geladen werden. Allegmeine Definitons Gunrdlage für das Design des Dokuments


\usepackage{iftex}
\ifPDFTeX
	\usepackage[utf8]{inputenc}
	\usepackage[T1]{fontenc}
\fi	
\usepackage{lmodern}
\usepackage[export]{adjustbox}
\usepackage{scrlayer-scrpage}
\usepackage{graphicx}%damit keine bilddateien gebraucht werden
\usepackage[dvipsnames]{xcolor}
\usepackage{array}
\usepackage{url} %für Pfaddarstellung
\urlstyle{same} %Für Pfaddarstellung
\usepackage[sfdefault]{ClearSans} % Use the Clear Sans font (sans serif)
\usepackage{tabularx}
\usepackage{colortbl}
\usepackage{textcomp}
\usepackage{longtable}
\usepackage{bmpsize}
\usepackage{calc}
\usepackage{array}
\usepackage[ngerman,english]{babel}



% Absatzabstand einstellen
\setparsizes{1em}{0.5\baselineskip plus 0.5\baselineskip}{1em plus 1fil}


%------------------------------------------------
%  Hinweise
%------------------------------------------------
% Für Pfade kann entweder \path oder \url verwendet werden

%------------------------------------------------
%  Keine Ahung warum das hier ist
%------------------------------------------------
\clearpairofpagestyles
%------------------------------------------------
%  Hintergrundbild
%------------------------------------------------
\DeclareNewLayer[
align=tl,
voffset=5mm,
hoffset=5mm,
width=\paperwidth-10mm,
background,
contents={
	\hfill\raisebox{-.5\totalheight}{\includegraphics[width=1cm]{pics/logo}}%bei echtem logo nur breite angeben
	\\[3pt]
}
]{headline}
\KOMAoptions{footsepline=1pt,footwidth=\dimexpr\paperwidth-10mm\relax}
\setkomafont{footsepline}{\color{lightgray}}
\ifoot{Marc Weidner - Schillerstrasse 34 - 88630 Pfullendorf}
\ofoot{\pagemark}
\renewcommand{\labelitemi}{\raisebox{.35\baselineskip}{\rule{2.5pt}{2.5pt}}}
\renewcommand{\labelitemii}{\raisebox{.35\baselineskip}{\rule{2.5pt}{2.5pt}}}
\renewcommand{\labelitemiii}{\raisebox{.35\baselineskip}{\rule{2.5pt}{2.5pt}}}
%------------------------------------------------
%  Seitenstile und Titelseite
%------------------------------------------------
\AddLayersToPageStyle{scrheadings}{headline}
\usepackage{blindtext}
\usepackage[top=2cm]{geometry}
\definecolor{schmuckfarbe}{cmyk}{0,0,0,0}
\addtokomafont{section}{\color{Blue}}%farbton anpassen
\addtokomafont{subsection}{\color{Blue}}%farbton anpassen
\addtokomafont{subsubsection}{\color{Blue}}%farbton anpassen
%------------------------------------------------
%  Definiton der Tabellenform für die Texte Bild Links Text Rechts
%------------------------------------------------
%\newenvironment{old}[1]{
%	\begin{minipage}[c]{\dimexpr.5\linewidth-\tabcolsep\relax}
%	\strut\\[-\baselineskip]#1
%\end{minipage}\hspace{2\tabcolsep}\begin{minipage}[c]{\dimexpr.5\linewidth-\tabcolsep\relax}}{\end{minipage}}
\newenvironment{eintrag}[1]{
	\par\medskip\noindent%
	\begin{minipage}[t]{\dimexpr.5\linewidth-10\tabcolsep\relax}
		\strut\\[-\baselineskip]#1
	\end{minipage}\hspace{2\tabcolsep}\begin{minipage}[t]{\dimexpr.6\linewidth-\tabcolsep\relax}}{\end{minipage}}
%------------------------------------------------
%  Für Icons links und neben drann die Beschreibung
%------------------------------------------------
\newenvironment{icontext}[1]{
	\par\medskip\noindent%
	\begin{minipage}[t]{\dimexpr.5\linewidth-30\tabcolsep\relax}
		\strut\\[-\baselineskip]#1
	\end{minipage}\hspace{2\tabcolsep}
\begin{minipage}[t]{\dimexpr1\linewidth-\tabcolsep\relax}}{\end{minipage}}