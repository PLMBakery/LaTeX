\documentclass{scrartcl}

% Deutsche Silbentrennung
\usepackage[ngerman]{babel}
\usepackage[T1]{fontenc}
\usepackage[utf8]{inputenc}

% Seitenränder anpassen
\usepackage{geometry}
\geometry{a4paper, left=25mm, right=25mm, top=25mm, bottom=25mm}

% Header und Footer
\usepackage{fancyhdr}
\pagestyle{fancy}
\fancyhf{}
\rhead{\thepage}
\lhead{\leftmark}
\renewcommand{\headrulewidth}{0.5pt}
\renewcommand{\footrulewidth}{0pt}

% Teamcenter Version, ITL VERSION, NX VERSION
\newcommand{\teamcenterVersion}{\textbf{Teamcenter Version: }\textit{Insert Teamcenter Version Here}}
\newcommand{\itlVersion}{\textbf{ITL VERSION: }\textit{Insert ITL Version Here}}
\newcommand{\nxVersion}{\textbf{NX VERSION: }\textit{Insert NX Version Here}}
\newcommand{\inc}{\textbf{INC Number: }\textit{Insert NINC Numb here}}
% Paket für Blindtext
\usepackage{blindtext}

\title{ \inc}
\author{Marc Weidner}

\begin{document}
	
	\maketitle
	% Zusätzliche Informationen
\vspace{15cm}
	\begin{itemize}
		\item \teamcenterVersion
		\item \itlVersion
		\item \nxVersion
	\end{itemize}

	\newpage
	\tableofcontents
	\newpage
	\section{Fehler 1: Beispiel}
	
	\subsection{Symptome}
	
	Hier beschreiben Sie die Symptome des Fehlers, die Sie festgestellt haben.
	
	\subsection{Analyse}
	
	Versuchen Sie, den Fehler zu analysieren. Überlegen Sie, was vor dem Auftreten des Fehlers passiert ist, und welche möglichen Ursachen es geben könnte.
	
	\subsection{Lösungsversuch}
	
	Beschreiben Sie alle Lösungsversuche, die Sie unternommen haben, um den Fehler zu beheben.
	
	\subsection{Ergebnis}
	
	Notieren Sie das Ergebnis Ihrer Lösungsversuche. War der Fehler behoben oder nicht? Wenn nicht, was waren die Gründe dafür?
	
	\section{Fehler 2: Beispiel}
	
	Wiederholen Sie den Prozess für jeden weiteren Fehler, den Sie identifizieren.
	
\end{document}
