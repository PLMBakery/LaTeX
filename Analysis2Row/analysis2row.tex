\documentclass[12pt, a4paper]{article}
\usepackage[utf8]{inputenc}
\usepackage[ngerman]{babel}
\usepackage{amsmath} % Für mathematische Formeln
\usepackage{graphicx} % Für Bilder
\usepackage{float} % Für Platzierung von Bildern und Diagrammen
\usepackage{caption}
\usepackage{multicol} % Für zweispaltiges Layout
\usepackage{geometry} % Für Seiteneinstellungen
\usepackage{fancyhdr} % Für Kopf- und Fußzeilen
\usepackage{hyperref} % Für Inhaltsverzeichnis-Verlinkungen

% Seiteneinstellungen
\geometry{a4paper, left=25mm, right=25mm, top=20mm, bottom=20mm}

% Kopf- und Fußzeilen
\pagestyle{fancy}
\fancyhf{}
\rhead{\thepage}

% Dokumentbeginn
\begin{document}
	
	% Deckblatt
	\begin{titlepage}
		\centering
		{\LARGE Meine wissenschaftliche Arbeit}\par
		\vspace{2cm}
		{\Large Vorname Nachname}\par
		\vspace{1cm}
		{\large Datum: \today}\par
		\vspace{3cm}
		\vfill
		{\large Zusammenfassung der Notizen}\par
	\end{titlepage}
	
	% Inhaltsverzeichnis
	\tableofcontents
	\newpage
	
	% Zweispaltiger Text
	\begin{multicols}{2}
		
		% Einleitung
		\section{Einleitung}
		Hier beginnt der Text deiner Arbeit. Dies ist ein Beispiel für einen zweispaltigen Abschnitt.
		
		% Bild einfügen
		\begin{figure}[H]
			\centering
			\includegraphics[width=0.45\textwidth]{example-image-a}
			\caption{Beispielbild}
			\label{fig:beispiel}
		\end{figure}
		
		% Mathematische Formel
		\section{Mathematische Formeln}
		Hier ist eine Beispielgleichung:
		\begin{equation}
			E = mc^2
		\end{equation}
		
		% Diagramm einfügen
		\section{Diagramme}
		Diagramme können als Bilder oder mit einem Tool wie TikZ eingefügt werden. Hier ist ein einfaches Beispiel:
		
		\begin{figure}[H]
			\centering
			\includegraphics[width=0.45\textwidth]{example-image-b}
			\caption{Beispieldiagramm}
			\label{fig:diagramm}
		\end{figure}
		
		% Zusammenfassung
		\section{Zusammenfassung}
		Hier kommt die Zusammenfassung deiner Notizen hin.
		
	\end{multicols}
	
\end{document}
