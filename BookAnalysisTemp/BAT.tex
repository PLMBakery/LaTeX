\documentclass[12pt, a4paper]{article}
\usepackage[utf8]{inputenc}
\usepackage[ngerman]{babel}
\usepackage{amsmath} % Für mathematische Formeln
\usepackage{graphicx} % Für Bilder
\usepackage{float} % Für Platzierung von Bildern und Diagrammen
\usepackage{caption}
\usepackage{multicol} % Für zweispaltiges Layout
\usepackage{geometry} % Für Seiteneinstellungen
\usepackage{fancyhdr} % Für Kopf- und Fußzeilen
\usepackage{hyperref} % Für Inhaltsverzeichnis-Verlinkungen
\usepackage{wrapfig}

% just to make this example:
\usepackage{lipsum}
\usepackage{mwe}
% Seiteneinstellungen
\geometry{a4paper, left=25mm, right=25mm, top=20mm, bottom=20mm}
% Kopf- und Fußzeilen
\pagestyle{fancy}
\fancyhf{}
\rhead{\thepage}
% Dokumentbeginn
\begin{document}
	
% Deckblatt
	\begin{titlepage}
		\centering
		{\LARGE Meine wissenschaftliche Arbeit}\par
		\vspace{2cm}
		{\Large Vorname Nachname}\par
		\vspace{1cm}
		{\large Datum: \today}\par
		\vspace{3cm}
		\vfill
		{\large Zusammenfassung der Notizen}\par
\end{titlepage}
			
% Inhaltsverzeichnis
	\tableofcontents
	\newpage
	
	
% Mathematische Formel
\section{Mathematische Formeln}
Hier ist eine Beispielgleichung:
\begin{equation}
	E = mc^2
\end{equation}

\section{Two Pictures}

\begin{figure}[h]
	\includegraphics[width=.45\textwidth]{example-image-a}
	\hfill
	\includegraphics[width=.45\textwidth]{example-image-b}
\end{figure}

\section{Text besides Figure}

\lipsum[1]

\begin{wrapfigure}{l}{0cm}
	\includegraphics[width=.3\textwidth]{example-image}
\end{wrapfigure}

\lipsum[1]
\end{document}